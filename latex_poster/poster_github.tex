%%%%%%%%%%%%%%%%%%%%%%%%%%%%%%%%%%%%%%%%%
% Jacobs Landscape Poster
%%%%%%%%%%%%%%%%%%%%%%%%%%%%%%%%%%%%%%%%%

\documentclass[final, hyperref={colorlinks,linkcolor=[RGB]{20,50,100},urlcolor=[RGB]{20,50,100}}]{beamer}

\usepackage{float}
\usepackage[size=a0,orientation=portrait,scale=1.24]{beamerposter} % Use the beamerposter package for laying out the poster

\usetheme{confposter} % Use the confposter theme supplied with this template

\usepackage{comment}

\usepackage[absolute,overlay]{textpos} %use showboxes option for seeing where the boxes are put
\usepackage{multicol}
\usepackage{booktabs}
\usepackage{setspace}
\usepackage{adjustbox}

\newlength{\sepwid}
\newlength{\onecolwid}
\newlength{\twocolwid}
\newlength{\threecolwid}
\setlength{\paperwidth}{841mm} 
\setlength{\paperheight}{1189mm} 

% 2-column layout:
% 2*onecolwid + 3*sepwid = \paperwidth
\setlength{\sepwid}{0.005\paperwidth} % Separation width (white space) between columns
\setlength{\onecolwid}{0.4925\paperwidth} % Width of one column
\setlength{\twocolwid}{0.99\paperwidth}    % Useful for full-width content (e.g. logo band)

\usepackage{graphics}  % Required for including images
\graphicspath{{poster_figures/}}
\usepackage{mwe}       % For example-image-a/b/c placeholders
\usepackage{lipsum}

\usepackage{booktabs} % Top and bottom rules for tables
\newcommand{\pl}{$\boldsymbol{+}$}
\newcommand{\mi}{$\boldsymbol{-}$}

\usepackage[font=footnotesize,labelfont=bf]{caption}

%----------------------------------------------------------------------------------------
%   TITLE SECTION 
%----------------------------------------------------------------------------------------

\title{Preliminary validation of satellite chlorophyll-{a} products using inline data from the Ocean Training Course 2025 (OTC25)} % Poster title

\author{\underline{Mathurin Choblet}$^{1}$, Sejal Pramlall$^{2}$, with support from the ESA OTC 2025 crew} % Author(s)

%\institute{Contact: mathurin@choblet.com \\[0.3em] \normalsize{$^{1}$MAST, University of Liège, Belgium, $^{2}$Marine Optics Laboratory, University of Bergen, Norway.$ }}
\institute{Contact: mathurin@choblet.com | \normalsize{$^{1}$MAST, University of Liège, Belgium; $^{2}$Marine Optics Laboratory, University of Bergen, Norway. }}

%Department of Physics and Technology,

%----------------------------------------------------------------------------------------
% COLOR, BOX
%----------------------------------------------------------------------------------------

\usepackage[many]{tcolorbox}

\tcbset{
  colback=White,
  arc=1mm,
  fontupper=\sffamily\normalsize,
  colframe=jblue,
  halign title=center,
  fonttitle=\bfseries\large\scshape,
  boxrule=3pt,
  after title=\vspace{8pt},
  before title=\vspace{8pt},
  before skip=-1cm
}

\newtcolorbox{InnerBox}[1][]{%
    enforce breakable,
    enhanced,
    colback=lightgray,
    colframe=white,
    #1
}%

\setbeamercolor{block title}{fg=white, bg=jblue}
\setbeamercolor{block body}{bg=white,fg=black}

%----------------------------------------------------------------------------------------

\begin{document}

\begin{frame}[t] % The whole poster is enclosed in one beamer frame
\large

\vspace{-1cm}

%----------------------------------------------------------------------------------------
% MAIN TWO-COLUMN LAYOUT
%----------------------------------------------------------------------------------------

\begin{columns}[t]
  \begin{column}{\sepwid}\end{column}

  %========================
  % LEFT COLUMN
  %========================
  \begin{column}{\onecolwid}

  %\begin{tcolorbox}[adjusted title=0. Objectives]
\begin{tcolorbox}[
%  title={\raisebox{-0.3em}{\includegraphics[height=1.3em]{target_picto.png}} Objectives}
title={\smash{\raisebox{-0.3em}{\includegraphics[height=1.3em]{target_picto_thick.png}}}\ Objectives}
]



\small
%\begin{enumerate}
\begin{itemize}
  \setlength\itemsep{0.5em}
  \item Characterize inline Chl-{a} variability along the Tromsø–Nice transect.
  \item Evaluate global and regional L3 and L4 satellite Chl-{a} products against inline matchups.
  \item Assess the added value of gap-filled and regional high-resolution OLCI products.
  \item Lay the groundwork for future HPLC and fluorometric Chl-{a} calibration of the inline system.
%\end{enumerate}
\end{itemize}

    \end{tcolorbox}
     \vspace{1cm}
    % 1. Cruise track & inline chl-a (big map figure)
    %\begin{tcolorbox}[adjusted title= Statsraad Lehmkuhl Cruise track and Inline CHL-{a}]
    \begin{tcolorbox}[
%  title={\raisebox{-0.3em}{\includegraphics[height=1.3em]{target_picto.png}} Objectives}
title={\smash{\raisebox{-0.3em}{\includegraphics[height=1.3em]{ship_picto2.png}}}\ Statsraad Lehmkuhl cruise track}
]
      \small
      %\lipsum[1][1-4]

      \begin{figure}
        \centering
        %\includegraphics[width=\textwidth,height=65cm]{example-image-a}
        \includegraphics[width=\textwidth]{map_norwegian_poster.png}
   
        \includegraphics[width=\textwidth]{map_northatlantic_poster.png}
        \includegraphics[width=\textwidth]{map_med_poster.png}
        %\caption*{\small Placeholder for multi-panel map showing cruise track, inline chlorophyll-{a}, satellite coverage and CTD stations.}
      \end{figure}
    \end{tcolorbox}

    \vspace{0.2cm}

\vspace{0.5cm}
\begin{tcolorbox}[
title={\smash{\raisebox{-0.55em}{\includegraphics[height=1.5em]{acs_picto_white.png}}}\smash{\raisebox{-0.55em}{\includegraphics[height=1.5em]{ctd_picto_white.png}}}\smash{\raisebox{-0.5em}{\includegraphics[height=1.3em]{satellite_picto_thick.png}}}\ Data and matchup protocol \smash{\raisebox{-0.6em}{\includegraphics[height=1.7em]{puzzle_pictogram_thick.png}}}}
]

%\begin{tcolorbox}[adjusted title=2. Data and matchup protocol]
\small
\begin{columns}[T,totalwidth=\textwidth]
  %---------------- LEFT: IN-SITU
  \begin{column}{0.48\textwidth}
      \vspace{1em}
    \textbf{Inline AC-S}\\[-0.2em]
    \begin{itemize}
      \setlength\itemsep{0.2em}
      \item WET Labs spectral inline absorption system
      \item Chl-{a} line height from $a_{676}$ \footnotesize{(Boss et al. 2013)}
      \item \small{Continuous underway sampling (at 2 m depth) along the cruise track}
    \end{itemize}

    \vspace{0.4em}
    \textbf{Satellite Chl-{a} products (L3 and L4)}\\[-0.2em]
    \begin{itemize}
      \setlength\itemsep{0.2em}

      \item Global, 4 km resolution:\\ PACE, MODIS-A, OC-CCI, GC-MY, GC-L4
      \item Regional, 1 km resolution: \\ GC-ATL and GC-MED-L3
      \item OLCI, 300 m resolution:\\ GC-OLCI, GC-MED-OLCI
     % \item All products have daily resolution
    \end{itemize}

  \end{column}

  %---------------- RIGHT: SATELLITE + MATCHUPS (MORE COMPACT)
  \begin{column}{0.48\textwidth}
      \vspace{1em}
    \textbf{CTD Fluorometer}\\[-0.2em]
    \begin{itemize}
      \setlength\itemsep{0.2em}
      \item SBE 19plus V2 with WET Labs ECO
\item Dark and non-photochemical quenching correction applied \footnotesize{(Xing et al. 2012)}

    \end{itemize}

    \vspace{0.4em}
    \textbf{Satellite-Inline Matchup}\\[-0.2em]
    \begin{itemize}
      \setlength\itemsep{0.2em}
      %\item Time: inline data restricted to 10:00–14:00
      \item Inline data restricted to 10:00–14:00 and bin-averaged to spatial satellite resolution
      \item Satellite mean in a $3\times3$ box centered around the bin-averaged inline data point
      \item Retain satellite value if the coefficient of variation is $<$ 0.15
      %\item Quality Control: retain filtered mean if coefficient of variation $<$ 0.15
      \item Procedure following Bailey \& Werdell 2006, Boss et al. 2013, Brewin et al. 2016
     % \item Resulting matched dataset differs for each product
    \end{itemize}
  \end{column}
\end{columns}

    \end{tcolorbox} 
  \end{column}

  \begin{column}{\sepwid}\end{column}

  %========================
  % RIGHT COLUMN
  %========================
  \begin{column}{\onecolwid}


    % 3. Key messages box (top-right, easy to skim) 
    %\begin{tcolorbox}[adjusted title=3. Key results]

\begin{tcolorbox}[
title={\smash{\raisebox{-0.5em}{\includegraphics[height=1.5em]{lightbulb_thick.png}}}\ Key results}
]

      \small
\begin{itemize}
  \setlength\itemsep{0.4em}
  \item Continuous underway AC-S sampling yields a rich matchup dataset with a large Chl-{a} dynamic range
        and a clear pattern of higher values at high latitudes and near coasts.
  \item All global satellite products show compressed Chl-{a} dynamic range and similar performance.
        OC-CCI performs best overall; for this transect we see no notable gain from higher-resolution.
  \item In the Mediterranean, global algorithms overestimate low Chl-{a}, while regional GC-MED products
        substantially reduce this bias as reported in the literature (e.g. \footnotesize{Volpe et al. 2019}\small).
  \item \small{CTD fluorescence yields systematically higher Chl-{a} relative to the inline system.}
\end{itemize}

    \end{tcolorbox}

    \vspace{0.2cm}

    % 4. Global product comparison
    %\begin{tcolorbox}[adjusted title=4. Global product comparison]
    \vspace{0.1cm}
\begin{tcolorbox}[
title={\smash{\raisebox{-0.5em}{\includegraphics[height=1.4em]{glob_picto_thick.png}}}\ Global product comparison}
]

      \small
      %\lipsum[3][1-3]

      \begin{figure}
        \centering
        %\includegraphics[width=40cm,height=25cm]{stats_global_poster.png}
        \includegraphics[width=\textwidth]{stats_global_poster.png}
        %\caption*{\small Placeholder for grid of scatter plots with inline statistics (N, $R^2$, RMSE, bias, slope).}
      \end{figure}
    \end{tcolorbox}

    \vspace{0.2cm}

    % 5. Regional performance (e.g. Mediterranean)
    %\begin{tcolorbox}[adjusted title=5. Regional performance in the Mediterranean Sea]
    \vspace{0.1cm}
\begin{tcolorbox}[
title={\smash{\raisebox{-0.5em}{\includegraphics[height=1.4em]{lookingglass_picto.png}}}\ Regional performance in the Mediterranean Sea}
]
      \small

      \begin{figure}
        \centering
        \includegraphics[width=\textwidth]{stats_med_sea_poster.png}
        %\caption*{\small Placeholder for regional comparison figure (e.g.\ Mediterranean global vs regional products).}
      \end{figure}
    \end{tcolorbox}

    \vspace{0.2cm}

    % 6. CTD–inline comparison + Outlook + QR codes
    %\begin{tcolorbox}[adjusted title=6. CTD--inline comparison and outlook]
    
\vspace{0.1cm}

\begin{tcolorbox}[
  title={\smash{\raisebox{-0.85em}{\includegraphics[height=2.2em]{telescop_picto.png}}}\ CTD--inline comparison and outlook}
]
  \small

  \begin{columns}[T,totalwidth=\linewidth]

    %----------------------------------------------------------
    % LEFT COLUMN: two figures stacked
    %----------------------------------------------------------
    \begin{column}{0.4\linewidth}

      \begin{figure}
        \centering
        \includegraphics[width=0.74\linewidth]{ctd_vs_inline_poster.png}
        %\caption*{\small Main CTD vs inline chlorophyll-a comparison.}
      \end{figure}
    \end{column}

    %----------------------------------------------------------
    % RIGHT COLUMN: bullets + two QR codes
    %----------------------------------------------------------
    \begin{column}{0.6\linewidth}
      \small

      \begin{itemize}
        \setlength\itemsep{0.4em}
        \item Calibrate the inline Chl-{a} record using discrete fluorometric and HPLC measurements.
        \item Revisit the CTD–inline-satellite comparison after calibration.
        \item Extend validation to lower-level products and make explicit use of quality flags.
        \item Design more controlled inter-product comparisons (e.g.\ equal-$N$ matchups).
        \item Consolidate the OTC25 dataset and code as an open benchmark for future OTC cohorts.
      \end{itemize}

      \vspace{0.3em}

      % QR codes (appendix-like content, but within same box)

   \begin{center}
        \textbf{\underline{Supplements}}\quad
        \begin{minipage}{0.23\linewidth}
          \centering
          \includegraphics[width=0.85\linewidth]{qr-code-drive}
          \\[-0.2em]
          \scriptsize Figures
        \end{minipage}
        \hspace{0.02\linewidth}
        \begin{minipage}{0.23\linewidth}
          \centering
          \includegraphics[width=0.85\linewidth]{qrcode_github}
          \\[-0.2em]
          \scriptsize Code repository
        \end{minipage}
      \end{center}

    \end{column}

  \end{columns}
\end{tcolorbox}

  \end{column}

  \begin{column}{\sepwid}\end{column}
\end{columns}

%----------------------------------------------------------------------------------------
% LOGO BAND
%----------------------------------------------------------------------------------------

%\vspace{-0.3cm}
\begin{beamercolorbox}[wd=35in,colsep=0.15cm]{cboxb}\end{beamercolorbox}

\begin{columns}
  \begin{column}{\sepwid}\end{column}

  \begin{column}{\onecolwid}
    \begin{figure}
      \raggedright
      % Replace example-image-* with actual logos (e.g. FNRS2, uliege, hereon, neccton)
      \includegraphics[height=4.5cm]{FNRS2.png}
      \hspace*{2cm}
      \includegraphics[height=4.5cm]{uliege.png}
      \hspace*{2cm}
      \includegraphics[height=5cm]{oneocean.png}
      \hspace*{1cm}
\adjustbox{height=4.5cm,trim={0.15\width} {0.20\height} {0.15\width} {0.20\height},clip}{%
  \includegraphics{ESA_logo_2020_Deep.png}%
}
    \end{figure}
  \end{column}

    \begin{column}{\onecolwid}
    \small{
      \textbf{Acknowledgments:} 
This work was supported by ESA (contract 4000145069/24/NL/IB/ar). Travel and accommodation were funded by the FRS-FNRS (grant 1.E.007.25F). We thank ESA for organizing the course and providing an excellent learning environment, and the crew of the Statsraad Lehmkuhl for their professional management of the voyage.
    }

  \end{column}

  \begin{column}{\sepwid}\end{column}
\end{columns}

\end{frame}

%----------------------------------------------------------------------------------------
% OLD / EXTRA MATERIAL (UNCHANGED, KEPT IN COMMENT)
%----------------------------------------------------------------------------------------


\end{document}
